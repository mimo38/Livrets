\documentclass[%
fontsize=14%
,a4paper%
,DIV=15%
]{scrartcl}
%scrartcl



\usepackage{gredocument}
\usepackage{adaptateur}
\usepackage{psaume}
\pagestyle{empty}
\title{\centrer{\huge{Veillée pascale}}}
\author{\includegraphics[height=10cm]{PâquesN-B.jpg}}
\date{la nuit de la Résurrection\\ avec baptêmes d'adultes}

\makeindex
        \definecolor{rubrum}{rgb}{.6,0,0}
        \def\rubrum{\color{rubrum}}%%%%%%%mettre"\def\rubrum{\color{rubrum}}" pour avoir le texte adéquat en rouge
        \def\nigra{\color{black}}
            %\redlines
            \definecolor{gregoriocolor}{rgb}{.6,0,0}
        %
        %\let\red\rubrum
        \newcommand{\rep}[2]{\versio{\R \textbf{#1}}{\R \textbf{#2}}}
        \newcommand{\vers}[2]{\versio{\V {#1}}{\V {#2}}}


%%%%%%%%%%%%%%%%%%%%%%%%%%%%%%%%%%%%%%%%%%%%%%%%%%%%%%%%%%%%%%%%%%%%%%%%%%%%%%%%%%%%%%%%%%%%%%%%%%%%
\begin{document}
\newfontfamily\lettrines[Scale=1.3]{LettrinesPro800}
    \def\gretextformat#1{{\fontsize{\taillepolice}{\taillepolice}\selectfont #1}}
    \def\greinitialformat#1{{\lettrines #1}}
    
\rubrica {Tous se tiennent debout et portent les cierges allumés à la main.}

\lettrine{E}{n cette nuit très sainte}, frères bien-aimés, la sainte Eglise notre
mère, au souvenir de notre Seigneur Jésus-Christ mort et reposant
au tombeau, veille avec amour, et célèbre sa résurrection glorieuse
avec une joie débordante.
Or, selon l'enseignement de Saint Paul ( Rom. 6, 4-11 « nous avons
été mis au tombeau avec le Christ par le baptême qui nous plonge dans
sa mort; et, de même que le Christ est ressuscité des morts, nous devons,
nous aussi, vivre d'une vie nouvelle; nous le savons, le vieil homme
que nous étions a été crucifié avec le Christ pour que, désormais, nous
ne soyons plus esclaves du péché. Aussi, prenons conscience que nous
sommes morts au péché et vivants pour Dieu, dans le Christ Jésus notre
Seigneur!»
C'est pourquoi, frères bien-aimés, après avoir terminé l'entraînement
du carême, renouvelons les engagements du saint baptême, par lesquels
autrefois nous avons renoncé à Satan et aux actes qu'il inspire, ainsi
qu'au monde qui est ennemi de Dieu, et nous nous sommes engagés
à servir Dieu fidèlement dans la sainte Eglise catholique.

LE CÉLÉBRANT : Renoncez-vous à Satan ?

Tous : Nous renonçons.

LE CÉLÉBRANT : Renoncez-vous à toutes ses oeuvres ?

Tous : Nous renonçons.

LE CÉLÉBRANT : Renoncez-vous à toutes ses séductions ?

Tous : Nous renonçons.

LE CÉLÉBRANT : Croyez-vous en Dieu le Père tout-puissant, créateur
du ciel et de la terre?

Tous : Nous croyons.

LE CÉLÉBRANT : Croyez-vous en Jésus-Christ son Fils unique, notro
Seigneur, qui est né et qui a souffert la Passion ?

Tous : Nous croyons.

LE CÉLÉBRANT : Croyez-vous aussi en l'Esprit-Saint, à la sainte Eglise.
catholique, à la communion des Saints, à la rémission
des péchés, à la résurrection de la chair et à la vie
éternelle ?

Tous : Nous croyons.

LE CÉLÉBRANT : Et maintenant, tous ensemble, prions Dieu, comme
notre Seigneur Jésus·Christ nous a enseigné à prier : 

Tous : Notre Père ...

Le CÉLÉBRANT : Et que Dieu tout-puissant, Père de notre Seigneur
Jésus-Christ, qui nous a fait renaître par l'eau et
l'Esprit-Saint; et qui nous a accordé la rémission
de tout péché, nous garde encore par sa grâce dans le
Christ Jésus notre Seigneur, pour la vie éternelle.

Tous : AMEN!

\rubrica{Le célébrant asperge alors les fidèles avec l'eau bénite recueillie, comme
on l'a dit plus haut, pendant la bénédiction de l'eau baptismale}
\end{document}