\documentclass[%
fontsize=10%
,a5paper%
,DIV=15%
]{scrartcl}
%scrartcl

\usepackage{gredocument}
\usepackage{psaume}

\title{\centrer{Sacrement de Mariage suivi de la messe}}
\author{}
\date{}

\makeindex
\definecolor{rubrum}{rgb}{.6,0,0}
\def\rubrum{\color{black}}%%%%%%%mettre"\def\rubrum{\color{rubrum}}" pour avoir le texte adéquat en rouge
\def\nigra{\color{black}}
%    \redlines
%    \definecolor{gregoriocolor}{rgb}{.6,0,0}
%
%\let\red\rubrum

\newcommand{\maversio}[3][1]{%
	{
	\parindent=0pt
	\begin{paracol}{2}
	\ensurevspace{#1\baselineskip}
	\switchcolumn\switchcolumn*
 	 #2\looseness=-1 
	\switchcolumn
	 #3\looseness=-1 
	\end{paracol}%
	
	}}
\begin{document}

	
\titre{\selectfont\lettrinesD{Le Sacrement de Mariage}}

\titre{Veni Creator}
\vspace{0.5cm}
\rubrique{On se met à genoux pour la première strophe.}
\traduction{1. Venez, Esprit-Saint Créateur, dans les âmes de vos fidèles : comblez de la grâce d'en haut les c\oe urs que vous avez créés.}

\partition{Hymne}{VeniCreator}{}
\input{Hymne/VeniCreator-Couplets}

\end{document}