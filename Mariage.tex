\documentclass[%
fontsize=10%
,a5paper%
,DIV=15%
]{scrartcl}
%scrartcl

\usepackage{gredocument}
\usepackage{psaume}

\title{\centrer{Sacrement de Mariage suivi de la messe}}
\author{}
\date{}

\makeindex
\definecolor{rubrum}{rgb}{.6,0,0}
\def\rubrum{\color{black}}%%%%%%%mettre"\def\rubrum{\color{rubrum}}" pour avoir le texte adéquat en rouge
\def\nigra{\color{black}}
%    \redlines
%    \definecolor{gregoriocolor}{rgb}{.6,0,0}
%
%\let\red\rubrum

\newcommand{\maversio}[3][1]{%
	{
	\parindent=0pt
	\begin{paracol}{2}
	\ensurevspace{#1\baselineskip}
	\switchcolumn\switchcolumn*
 	 #2\looseness=-1 
	\switchcolumn
	 #3\looseness=-1 
	\end{paracol}%
	
	}}
\begin{document}

	
\titre{\selectfont\lettrinesD{Le Sacrement de Mariage}}

\titre{Veni Creator}
\vspace{0.5cm}
\rubrique{On se met à genoux pour la première strophe.}
\traduction{1. Venez, Esprit-Saint Créateur, dans les âmes de vos fidèles : comblez de la grâce d'en haut les c\oe urs que vous avez créés.}

\partition{Hymne}{VeniCreator}{}
\noindent
\traduire{%
2.~Qui díceris Paráclitus,\\%
Altíssimi donum Dei,\\%
Fons vivus, ignis, cáritas\\%
Et spiritális únctio.}{%
2.~Vous qu'on appelle Consolateur,\\%
don du Dieu très-haut,\\%
source vive, feu, charité\\%
et onction spirituelle.\\%

}

\traduire{%
3.~Tu septifórmis múnere,\\%
Dig\textit{i}tus patérnæ déxteræ,\\%
Tu rite promíssum Patris,\\%
Sérmone ditans gúttura.}{%
3.~Vous l'Esprit aux sept dons,\\%
le doigt de Dieu,\\%
la promesse authentique du Père,\\%
qui mettez sa parole sur nos lèvres.\\%

}


\traduire{%
4.~Accénde lumen sénsibus,\\%
Infúnd\textit{e} amórem córdibus,\\%
Infírma nostri córporis\\%
Vírtute firmans pérpeti.}{%
4.~Éclairez nos esprits de votre lumière,\\%
mettez l'amour dans nos c\oe urs~;\\%
soutenez la faiblesse de notre corps\\%
par votre constante vigueur.\\%

}


\traduire{%
5.~Hostem repéllas lóngius,\\%
Pacémque dones prótinus~:\\%
Ductóre sic te pr\'ævio\\%
Vitémus omne nóxium.\\}{%
5.~Chassez l'ennemi loin de nous,\\%
donnez-nous sans retard la paix~;\\%
guidez-nous, et que sous votre conduite\\%
nous évitions tout mal.\\%
}

\traduire{%
6.~Per te sciámus da Patrem,\\%
Noscámus atque Fílium,\\%
Tequ\textit{e} utriúsque Spíritum\\%
Credámus omni témpore.\\}{%
6.~Faites-nous connaître le Père,\\%
faites-nous connaître le Fils,\\%
donnez-nous de toujours croire en vous\\%
qui êtes l'Esprit du Père et du Fils.\\%
}

\traduire{%
7.~Deo Patri sit glória,\\%
Et Fílio qu\textit{i} a mórtuis\\%
Surréxit, ac Paráclito,\\%
In sæculórum sǽcula.\\%
Amen\\}{%
7.~Gloire soit à Dieu le Père,\\%
et au Fils ressuscité des morts,\\%
et à l'Esprit consolateur\\%
dans les siècles des siècles.\\%
\ainsi\\%
}

\traduire{%
℣.~Emitte Spiritum tuum et creabúntur.}{%
℣.~Envoyez votre Esprit Seigneur et il se fera une création nouvelle.}

\traduire{\textbf{%
℟.~Et renovábis fáciem terræ.}}{\textbf{%
℟.~Et vous renouvellerez la face de la terre.}}

\traduire{%
Orémus.}{%
Prions.}

\traduire{%
Deus, qui corda fidélium Sancti Spíritus illustratióne docuísti : da nobis in eódem Spíritu recta sápere, et de eius consolatióne gaudére. \Perdominum \quitecum \peromnia%
}{%
Ô Dieu, qui avez éclairé les cœurs de vos fidèles par la lumière du Saint-Esprit, donnez-nous par ce même Esprit de comprendre et d'aimer ce qui est bien, et de jouir sans cesse de ses divines consolations. \Parjesus \quietant \siecles}
\amen


\end{document}